%%% This template was originally exported from LyX
%%% and modified manually later on.

\documentclass[polish]{dinbrief}
\renewcommand{\familydefault}{\sfdefault}
\usepackage[T1]{fontenc}
\usepackage[utf8]{inputenc}
\usepackage{array}

\makeatletter

%%%%%%%%%%%%%%%%%%%%%%%%%%%%%% LyX specific LaTeX commands.
\newcommand{\lyxline}[1][1pt]{%
  \par\noindent%
  \rule[.5ex]{\linewidth}{#1}\par}
%% Because html converters don't know tabularnewline
\providecommand{\tabularnewline}{\\}

%%%%%%%%%%%%%%%%%%%%%%%%%%%%%% Textclass specific LaTeX commands.
\newcommand{\areacode}{}

\makeatother

\usepackage{babel}

\begin{document}
\begin{flushleft}
\textbf{\Large Faktura VAT}
\par\end{flushleft}{\Large \par}

\begin{flushleft}
\emph{Numer:} <% @var invoice-id %>\\
\emph{Data wystawienia:} <% @var invoice-date-full %>\\
\emph{Data sprzedaży:} <% @var invoice-date-full %>
\par\end{flushleft}

\begin{flushleft}
\medskip{}

\par\end{flushleft}

\lyxline{\normalsize}

%%%
%%% This is the buyer-seller section.
%%%

\begin{flushleft}
\begin{tabular}{|l|l|l|}
\cline{1-1} \cline{3-3} 
\emph{Sprzedawca:} &  & \emph{Nabywca:}\tabularnewline
\cline{1-1} \cline{3-3} 
Grzymała Design &  & <% @var buyer-name %>\tabularnewline
Odyńca 7 m. 9 &  & <% @var buyer-address %>\tabularnewline
02-606 Warszawa &  & <% @var buyer-postcode %> <% @var buyer-city %>\tabularnewline
NIP: 521-174-36-40 &  & NIP: <% @var buyer-nip %>\tabularnewline
Konto: 30 1140 2004 0000 3802 1456 7893 &  & \tabularnewline
\cline{1-1} \cline{3-3} 
\end{tabular}
\par\end{flushleft}

\medskip{}

\lyxline{\normalsize}

%%%
%%% This is the item list section
%%%

\begin{flushright}
\medskip{}
\begin{tabular}{r>{\raggedleft}p{0.2\paperwidth}rr|r|r|r|r|}
lp. & nazwa & cena netto & \multicolumn{1}{r}{ilość} & \multicolumn{1}{r}{wartość netto} & \multicolumn{1}{r}{VAT} & \multicolumn{1}{r}{kwota VAT} & \multicolumn{1}{r}{wartość brutto}\tabularnewline
\hline 
\multicolumn{1}{|r|}{<number>} & \multicolumn{1}{>{\raggedleft}p{0.2\paperwidth}|}{<item-name>} & \multicolumn{1}{r|}{<item-net-price> zł} & <item-count> & <item-net-value> zł & <vat>\% & <vat-value> zł & <item-gross-value> zł\tabularnewline
\hline 
 &  &  & Łącznie: & <net-total> zł & X & <vat-total> zł & <gross-total> zł\tabularnewline
\cline{5-8} 
 &  &  &  & <22-net-total> zł & 22\% & <22-vat-total> zł & <22-gross-total> zł\tabularnewline
\cline{5-8} 
 &  &  &  & <7-net-total> zł & 7\% & <7-vat-total> zł & <7-gross-total> zł\tabularnewline
\cline{5-8} 
 &  &  &  & <3-net-total> zł & 3\% & <3-vat-total> zł & <3-gross-total> zł\tabularnewline
\cline{5-8} 
 &  &  &  & <zw-net-total> zł & zw. & <zw-vat-total> zł & <zw-gross-total> zł\tabularnewline
\cline{5-8} 
\end{tabular}
\par\end{flushright}

\medskip{}

%%%
%%% This is the footer with the payment spelled out in words and payment type/date
%%%


\begin{flushright}
Do zapłaty: <gross-total> zł.\\
Słownie: <words-gross-total>, <gross-total-cent>/100.\\
Forma płatności: <payment-form>.\\
Termin płatności: <payment-date>.
\par\end{flushright}
\end{document}
