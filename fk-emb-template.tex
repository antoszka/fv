%%% This template was originally exported from LyX
%%% and modified manually later on.
%%% It it a part of the Common Lisp fv personal invoicing program
%%% (c) 2010 by Antoni Grzymała

\documentclass[polish]{dinbrief}
\renewcommand{\familydefault}{\sfdefault}
\usepackage[T1]{fontenc}
\usepackage[utf8]{inputenc}
\usepackage{array}

\makeatletter

%%%%%%%%%%%%%%%%%%%%%%%%%%%%%% LyX specific LaTeX commands.
\newcommand{\lyxline}[1][1pt]{%
  \par\noindent%
  \rule[.5ex]{\linewidth}{#1}\par}
%% Because html converters don't know tabularnewline
\providecommand{\tabularnewline}{\\}

%%%%%%%%%%%%%%%%%%%%%%%%%%%%%% Textclass specific LaTeX commands.
\newcommand{\areacode}{}

\makeatother

\usepackage[polish]{babel}

\begin{document}
\begin{flushleft}
\textbf{\Large Faktura korygująca VAT}
\par\end{flushleft}{\Large \par}

\begin{flushleft}
\emph{Numer:} <%            @var corr-invoice-id %>\\
\emph{Do faktury:} <%       @var invoice-id %>\\
\emph{Data wystawienia:} <% @var corr-invoice-date-full %>\\
\emph{Data korekty:} <%     @var corr-invoice-date-full %>
\par\end{flushleft}

\begin{flushleft}
\medskip{}

\par\end{flushleft}

\lyxline{\normalsize}

%%%
%%% This is the buyer-seller section.
%%%

\begin{flushleft}
\begin{tabular}{|l|l|l|}
\cline{1-1} \cline{3-3} 
\emph{Sprzedawca:} &  & \emph{Nabywca:}\tabularnewline
\cline{1-1} \cline{3-3} 
<% @var seller-name %> &  & <% @var buyer-name %>\tabularnewline
<% @var seller-address %> &  & <% @var buyer-address %>\tabularnewline
<% @var seller-postcode %> <% @var seller-city %> &  & <% @var buyer-postcode %> <% @var buyer-city %>\tabularnewline
NIP: <% @var seller-nip %> &  & NIP: <% @var buyer-nip %>\tabularnewline
Konto: <% @var seller-account %> &  & \tabularnewline
\cline{1-1} \cline{3-3} 
\end{tabular}
\par\end{flushleft}

\medskip{}

\lyxline{\normalsize}
\begin{flushright}
\medskip{}

%%%
%%% This is OUT the item list section
%%%

\begin{tabular}{r>{\raggedleft}p{0.2\paperwidth}rr|r|r|r|r|}
lp. & nazwa & cena netto & \multicolumn{1}{r}{ilość} & \multicolumn{1}{r}{wartość netto} & \multicolumn{1}{r}{VAT} & \multicolumn{1}{r}{kwota VAT} & \multicolumn{1}{r}{wartość brutto}\tabularnewline
\hline

%%% This has to be looped across the removed item list:
<% (dolist (item-calculations (getf env :out-calculated-items))
	   (format t
		   "~{\\multicolumn\{1\}\{|r|\}\{~a\} & \\multicolumn\{1\}\{>\{\\raggedleft\}p\{0.2\\paperwidth\}|\}\{~a\} & \\multicolumn\{1\}\{r|\}\{~a zł\} & ~a & ~a zł & ~a\\ & ~a zł & ~a zł\\tabularnewline\\hline~}~%%~%"
		   item-calculations)) %>
\end {tabular}
\medskip {}

%%%
%%% This is OUT the item list section
%%%

\begin{tabular}{r>{\raggedleft}p{0.2\paperwidth}rr|r|r|r|r|}
lp. & nazwa & cena netto & \multicolumn{1}{r}{ilość} & \multicolumn{1}{r}{wartość netto} & \multicolumn{1}{r}{VAT} & \multicolumn{1}{r}{kwota VAT} & \multicolumn{1}{r}{wartość brutto}\tabularnewline
\hline

%%% This has to be looped across the inserted item list:
<% (dolist (item-calculations (getf env :in-calculated-items))
	   (format t
		   "~{\\multicolumn\{1\}\{|r|\}\{~a\} & \\multicolumn\{1\}\{>\{\\raggedleft\}p\{0.2\\paperwidth\}|\}\{~a\} & \\multicolumn\{1\}\{r|\}\{~a zł\} & ~a & ~a zł & ~a\\ & ~a zł & ~a zł\\tabularnewline\\hline~}~%%~%"
		   item-calculations)) %>
\end {tabular}
\medskip {}

%%% This is the summary section:

\begin{tabular}{r>{\raggedleft}p{0.2\paperwidth}rr|r|r|r|r|}
 &  &  & Łącznie: & <% @var    corr-net-total %> zł &    X & <% @var    corr-vat-total %> zł & <% @var    corr-gross-total %> zł\tabularnewline
\cline{5-8} 
 &  &  &  &         <% @var corr-23-net-total %> zł & 23\% & <% @var corr-23-vat-total %> zł & <% @var corr-23-gross-total %> zł\tabularnewline
\cline{5-8} 
 &  &  &  &         <% @var  corr-8-net-total %> zł &  8\% & <% @var  corr-8-vat-total %> zł & <% @var  corr-8-gross-total %> zł\tabularnewline
\cline{5-8} 
 &  &  &  &         <% @var  corr-5-net-total %> zł &  5\% & <% @var  corr-5-vat-total %> zł & <% @var  corr-5-gross-total %> zł\tabularnewline
\cline{5-8} 
 &  &  &  &         <% @var corr-zw-net-total %> zł &  zw. &                         0,00 zł & <% @var   corr-zw-net-total %> zł\tabularnewline
\cline{5-8} 
\end{tabular}
\par\end{flushright}

\medskip{}

%%%
%%% This is the footer with the payment spelled out in words and payment type/date
%%%


\begin{flushright}
Do zapłaty: <% @var corr-gross-total %> zł.\\
Słownie:    <% @var corr-words-gross-total %>, <% @var corr-gross-total-cent %>/100.\\
            <% @var corr-payment-form %> %%% Forma płatności.
\par\end{flushright}
\end{document}
